\documentclass[]{elsarticle} %review=doublespace preprint=single 5p=2 column
%%% Begin My package additions %%%%%%%%%%%%%%%%%%%
\usepackage[hyphens]{url}



\usepackage{lineno} % add
\providecommand{\tightlist}{%
  \setlength{\itemsep}{0pt}\setlength{\parskip}{0pt}}

\usepackage{graphicx}
\usepackage{booktabs} % book-quality tables
%%%%%%%%%%%%%%%% end my additions to header

\usepackage[T1]{fontenc}
\usepackage{lmodern}
\usepackage{amssymb,amsmath}
\usepackage{ifxetex,ifluatex}
\usepackage{fixltx2e} % provides \textsubscript
% use upquote if available, for straight quotes in verbatim environments
\IfFileExists{upquote.sty}{\usepackage{upquote}}{}
\ifnum 0\ifxetex 1\fi\ifluatex 1\fi=0 % if pdftex
  \usepackage[utf8]{inputenc}
\else % if luatex or xelatex
  \usepackage{fontspec}
  \ifxetex
    \usepackage{xltxtra,xunicode}
  \fi
  \defaultfontfeatures{Mapping=tex-text,Scale=MatchLowercase}
  \newcommand{\euro}{€}
\fi
% use microtype if available
\IfFileExists{microtype.sty}{\usepackage{microtype}}{}
\usepackage[left=3cm,right=3cm,top=3cm,bottom=3cm]{geometry}
\bibliographystyle{elsarticle-harv}
\usepackage{longtable}
\ifxetex
  \usepackage[setpagesize=false, % page size defined by xetex
              unicode=false, % unicode breaks when used with xetex
              xetex]{hyperref}
\else
  \usepackage[unicode=true]{hyperref}
\fi
\hypersetup{breaklinks=true,
            bookmarks=true,
            pdfauthor={},
            pdftitle={Edge effects in Alberta},
            colorlinks=false,
            urlcolor=blue,
            linkcolor=magenta,
            pdfborder={0 0 0}}
\urlstyle{same}  % don't use monospace font for urls

\setcounter{secnumdepth}{5}
% Pandoc toggle for numbering sections (defaults to be off)

% Pandoc citation processing
\newlength{\csllabelwidth}
\setlength{\csllabelwidth}{3em}
\newlength{\cslhangindent}
\setlength{\cslhangindent}{1.5em}
% for Pandoc 2.8 to 2.10.1
\newenvironment{cslreferences}%
  {}%
  {\par}
% For Pandoc 2.11+
\newenvironment{CSLReferences}[3] % #1 hanging-ident, #2 entry spacing
 {% don't indent paragraphs
  \setlength{\parindent}{0pt}
  % turn on hanging indent if param 1 is 1
  \ifodd #1 \everypar{\setlength{\hangindent}{\cslhangindent}}\ignorespaces\fi
  % set entry spacing
  \ifnum #2 > 0
  \setlength{\parskip}{#2\baselineskip}
  \fi
 }%
 {}
\usepackage{calc} % for calculating minipage widths
\newcommand{\CSLBlock}[1]{#1\hfill\break}
\newcommand{\CSLLeftMargin}[1]{\parbox[t]{\csllabelwidth}{#1}}
\newcommand{\CSLRightInline}[1]{\parbox[t]{\linewidth - \csllabelwidth}{#1}}
\newcommand{\CSLIndent}[1]{\hspace{\cslhangindent}#1}

% Pandoc header
\makeatletter \def\ps@pprintTitle{  \let\@oddhead\@empty  \let\@evenhead\@empty  \def\@oddfoot{\centerline{\thepage}} \let\@evenfoot\@oddfoot} \makeatother \usepackage{float} \floatplacement{figure}{H} \newcommand{\beginsupplement}{\setcounter{table}{0} \renewcommand{\thetable}{S\arabic{table}} \setcounter{figure}{0} \renewcommand{\thefigure}{S\arabic{figure}}} \usepackage{setspace} \linenumbers



\begin{document}
\begin{frontmatter}

  \title{Edge effects in Alberta}
    \author[University of Calgary]{Samuel V. J. Robinson\corref{1}}
   \ead{samuel.robinson@ucalgary.ca} 
    \author[University of Calgary]{Lan H. Nguyen}
   \ead{hoanglan.nguyen@ucalgary.ca} 
    \author[University of Calgary]{Paul Galpern}
   \ead{paul.galpern@ucalgary.ca} 
      \address[University of Calgary]{2500 University Drive NW, Calgary, AB}
      \cortext[1]{Corresponding Author}
  
  \begin{abstract}
  Abstract goes here
  \end{abstract}
   \begin{keyword} beetles; spiders; harvestmen;\end{keyword}
 \end{frontmatter}

\newpage
\doublespacing

\hypertarget{introduction}{%
\section{Introduction}\label{introduction}}

\begin{itemize}
\tightlist
\item
  Intensive agricultural production has increased over the last 100 years, and agricultural land now makes up over a third of ice-free land on Earth

  \begin{itemize}
  \tightlist
  \item
    This has allows increases in human population and increased (global) stability in production
  \item
    However, this is not without cost, as higher-diversity non-crops are converted to lower-diversity crops, resulting in loss of habitat and overall biodiversity of non-target organisms
  \item
    Maintaining both biodiversity and production in agroecosystems represents a seldom-considered goal of conservationists and agronomists, and hold the potential for win-win scenarios
  \item
    Key to this is the preservation of semi-natural land (SNL), which represents the interface between crops and non-crops within agroecosystems
  \end{itemize}
\item
  SNL in and around crops is important for both agricultural production and conservation

  \begin{itemize}
  \tightlist
  \item
    They are habitat for mobile organisms, and can therefore act as sources of ecosystem services such as pollination or pest control
  \item
    They also can create microclimate effects that reduce extreme temperature, trap moisture, and reduce wind speed
  \item
    Unfortunately, most of the research is concentrated in Europe, and they tend to be less-studied on other continents
  \item
    In particular, North American agroecosystems have larger fields, and different varieties and agronomic practices, all of which could negate effects of SNL
  \end{itemize}
\end{itemize}

\begin{itemize}
\tightlist
\item
  SNL may affect yields at intermediate distances, depending on the spatial scale at which ecosystem services operate

  \begin{itemize}
  \tightlist
  \item
    Edge effects cause low yields at the edge of crops because of sparse or late seedling emergence, poor microclimate, and competition with weeds
  \item
    At the same time, the centre of large fields will not receive ecosystem services if they decay with distance from edges
  \item
    For example, pollination services from central place foragers that nest in SNL but forage in crops drops rapidly with distance
  \item
    Therefore, yield may be maximized at intermediate distances, where the ecosystem services cancel out negative edge effects
  \item
    This suggests a ``goldilocks" field size, where negative edge effects are canceled out by ecosystem services
  \end{itemize}
\end{itemize}

\begin{itemize}
\tightlist
\item
  Studies of SNL effects on crop yield also suffer from limited scope (e.g.~few crop types) and small sample sizes, limiting inference and reducing generality (but see wheat study)

  \begin{itemize}
  \tightlist
  \item
    For this reason, large-scale precision yield data holds enormous promise for agronomy, as it allows
  \item
    However, its use is limited for several reasons: 1) Lack of standardized formats between equipment types, 2) Sensor calibration required for field-level accuracy, and 3) unfamiliarity with spatial statistics
  \item
    Ecosystem services can influence both the mean and variability of yield in agroecosystems
  \item
    Typically only averages (means) are considered, but higher stability (lower variance) in yield can also be valuable
  \item
    Size examples: wheat study from the UK
  \item
    There are few studies of yield variability (only those with large datasets), but precision yield data opens up the possibility of modeling within-field variability, as well as average yield
  \end{itemize}
\item
  In this paper we ask:

  \begin{enumerate}
  \def\labelenumi{\arabic{enumi}.}
  \tightlist
  \item
    How does crop yield change with distance from the edge of field?
  \item
    Does this depend on type of field edge?
  \item
    Is there an intermediate distance where yield is maximized or variance is minimized?
  \end{enumerate}
\end{itemize}

\hypertarget{methods}{%
\section{Methods}\label{methods}}

\hypertarget{data-collection}{%
\subsection{Data collection}\label{data-collection}}

\begin{itemize}
\tightlist
\item
  Precision yield data were collected directly from farmers across Alberta

  \begin{itemize}
  \tightlist
  \item
    Farmers were solicited for yield data through local agronomists, and we received 298 field-years of data from 5 growers across a total of 7 years (2014-2020)
  \item
    We converted data to a standard csv format using Ag Leader SMS
  \item
    72\% of the crop types where either wheat (\emph{Triticum aestivum}) or canola (\emph{Brassica napus}), two of the most common crops in rotation in Alberta
  \item
    The remaining crop types were poorly replicated in our sample, so we constrained our analysis to only field-years containing wheat (94) or canola (119)
  \item
    Individual fields contained between 1 to 5 years of data (mean: 2.7)
  \item
    Containing a total of 18.4 million data points
  \end{itemize}
\item
  Yield data is collected in rectangles of the same length as the data interval (distance = combine ground speed \(\times\) interval, typically 1 second) and the same width as the combine header (5-7 m)

  \begin{itemize}
  \tightlist
  \item
    We extracted the size of each polygon (m\(^2\)), dry yield (tonnes), and the spatial location, and the sequence of collection (1 - end of harvest)
  \item
    Because of the large number of yield rectangles per field (30-800 thousand), we used the centroid of each polygon as its location, treating areal data as point data
  \item
    Seeding and application rates were constant across fields, so we did not consider inputs in our analysis
  \item
    We used dry yield (tonnes of seed/hectare after accounting for crop moisture) as our measure of crop yield
  \item
    Due to the large number of data at each field, we sub-sampled to 50,000 data points per field to reduce computation time
  \end{itemize}
\item
  Field boundaries were automatically digitized using buffers from the yield data locations, then manually checked using satellite imagery from Google Earth and classified land cover data from AAFC

  \begin{itemize}
  \tightlist
  \item
    Crop boundaries are flexible, and often change yearly depending on planting and emergence conditions (e.g.~flooding during some years)
  \item
    Additionally, seminatural features often change yearly

    \begin{itemize}
    \tightlist
    \item
      Ephemeral wetlands are flooded during some years, but consist mainly of grasses during dry years
    \item
      Grass boundaries can change if fields are used for as haying or pasture during crop rotation
    \end{itemize}
  \item
    This makes accurate and consistent classification of field boundaries difficult
  \item
    We defined the following general categories for field boundaries:

    \begin{enumerate}
    \def\labelenumi{\arabic{enumi}.}
    \tightlist
    \item
      Standard: grassy field edge, staging yard, or road right-of-way (grassy strip typically 5-10m wide)
    \item
      Wetland: permanent wetland; borders are largely unchanged from year-to-year
    \item
      Shelterbelt: permanent windbreaks, shelterbelts, remnant forests, or shrublands
    \item
      Other crop: annual crop or pasture with little or no visible boundary between planted areas
    \item
      Bare: unplanted, fallow, flooded area, temporary wetland (only present for a single season), staging yard, oil and gas equipment, or road without a planted boundary
    \item
      Grassland: permanent seminatural grassland or pasture (not in rotation)
    \end{enumerate}
  \end{itemize}
\end{itemize}

\hypertarget{analysis}{%
\subsection{Analysis}\label{analysis}}

\begin{itemize}
\item
  At each field, we fit an additive model of the effect of boundary distance on crop yield while accounting for within-field spatial variation and temporal variation in the combine yield monitor

  \begin{itemize}
  \item
    Crop yield can vary within a field due to soil conditions, moisture, seeding rates, herbicide application, and previous agricultural practices
  \item
    Ground speed is extremely important to yield monitor accuracy (Arslan \& Colvin 2002), with low ground speed registering higher yields
  \item
    While sensor calibration can reduce combine-level bias (such as a combine recording consistently higher/lower yields across fields), this does not address sensor drift that occurs over time within fields
  \item
    This may be caused by sensors accumulating debris during harvest (pers. comm. Trent Clark), leading to changes in accuracy and bias over time
  \item
    To model this, we fit the following model to each field-year of data:
  \end{itemize}

  \begin{equation}
  \begin{split}
  sqrt(yield) \sim & Normal (\mu, \sigma)\\
  \mu = & Intercept + log(Polygon Size) + f(Distance from Edge_i, b=12)_i + \\
   & f(Easting, Northing, b=60) + f(Sequence, b=60) \\
  log(\sigma) = & Intercept + log(Polygon Size) + f(Distance from Edge, b=12) + \\
   & f(Easting, Northing, b=60) + f(Sequence, b=60)
  \end{split}
  \end{equation}

  \begin{itemize}
  \tightlist
  \item
    where:
  \end{itemize}

  \begin{enumerate}
  \def\labelenumi{\arabic{enumi}.}
  \tightlist
  \item
    Polygon Size = distance traveled \(\times\) width of header bar (\(m^2\))
  \item
    Distance from Edge = distance from field edge type \emph{i} (m)
  \item
    Easting, Northing = distance from centre of field (m)
  \item
    Sequence = order of harvest within field (1--N points)
  \item
    f(x,b) = penalized thin-plate regression spline, where x is the predictor and b is the number of basis dimensions
  \end{enumerate}
\item
  In addition to modeling edge effects (our variable of interest), this model also accounts for a) differences in combine speed, b) within-field spatial variation not related to edges, and d) shifts in combine accuracy during harvest

  \begin{itemize}
  \tightlist
  \item
    Spatial or temporal variation is typically modelled using a Gaussian Process Model (Kriging) or approximations such as Stochastic Partial Differential Equations (e.g.~INLA), but this was computationally infeasible with 50,000 data points per field
  \item
    Penalized splines offer a compromise, as they account for nonlinear ``wiggly" relationships in the same way as Gaussian processes but with substantially reduced computation time
  \item
    The number of basis dimensions was checked with the \emph{gam.check} function from \emph{mgcv}
  \item
    The relationship between polygon size (i.e.~ground speed) and yield was modeled with a log-linear relationship with a single slope term, as this closely matched smoothed versions
  \item
    All models were fit in \emph{R} using the \emph{mgcv} library (version 1.8.36, Wood 2017), and figures were created with \emph{ggplot2} and \emph{ggpubr} (versions 3.3.3, Wickham 2016; and 0.4.0, Kassambara 2020).
  \end{itemize}
\item
  To consider results from all field-level models, we fit models independent of each other, and ``overall'' smoothers were taken as averages of the field-level smoothers

  \begin{itemize}
  \tightlist
  \item
    However, this does not account for uncertainty in the field-level smoothers, so we used an approach similar to bootstrapping of hierarchical mixed effects models
  \end{itemize}

  \begin{enumerate}
  \def\labelenumi{\arabic{enumi}.}
  \tightlist
  \item
    Extract single posterior sample (\emph{rnorm} in R) of smoother parameters from each field-level model using coefficient estimates and standard error
  \item
    Use posterior sample to create new simulated smoother from each field
  \item
    Fit new model of simulated smoothers from all fields, and save this ``meta-smoother''
  \item
    Repeat 1000 times, and calculate coverage intervals (5-95\% percentiles) on saved meta-smoothers
  \end{enumerate}

  \begin{itemize}
  \tightlist
  \item
    This gives coverage intervals (CIs) for the ``average'' smoother while accounting for field-level variability
  \end{itemize}
\end{itemize}

\hypertarget{results}{%
\section{Results}\label{results}}

Results here

\hypertarget{discussion}{%
\section{Discussion}\label{discussion}}

Discussion here

\hypertarget{authors-contributions}{%
\section{Authors' contributions}\label{authors-contributions}}

Author's contribution

\hypertarget{acknowledgements}{%
\section{Acknowledgements}\label{acknowledgements}}

Funding for this research was provided by Ducks Unlimited Canada's Institute for Wetland and Waterfowl Research, the Alberta Canola Producers Commission, Manitoba Canola Growers Association, SaskCanola, the Alberta Biodiversity Monitoring Institute, and the Alberta Conservation Association.

\hypertarget{references}{%
\section*{References}\label{references}}
\addcontentsline{toc}{section}{References}

\hypertarget{refs}{}
\begin{CSLReferences}{1}{0}
\leavevmode\hypertarget{ref-arslan2002}{}%
Arslan, S. \& Colvin, T.S. (2002). An evaluation of the response of yield monitors and combines to varying yields. \emph{Precision Agriculture}, 3, 107--122.

\leavevmode\hypertarget{ref-kassambara2020}{}%
Kassambara, A. (2020). \emph{{ggpubr}: 'ggplot2' based publication ready plots}.

\leavevmode\hypertarget{ref-wickham2016}{}%
Wickham, H. (2016). \emph{{ggplot2}: Elegant graphics for data analysis}. Springer-Verlag New York.

\leavevmode\hypertarget{ref-wood2017}{}%
Wood, S.N. (2017). \emph{Generalized additive models: An introduction with {R}}. CRC press.

\end{CSLReferences}

\newpage

\hypertarget{appendix-a-supplementary-material}{%
\section*{Appendix A: Supplementary Material}\label{appendix-a-supplementary-material}}
\addcontentsline{toc}{section}{Appendix A: Supplementary Material}

Supplemental materials here


\end{document}

