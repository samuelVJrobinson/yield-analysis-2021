\documentclass[]{elsarticle} %review=doublespace preprint=single 5p=2 column
%%% Begin My package additions %%%%%%%%%%%%%%%%%%%
\usepackage[hyphens]{url}



\usepackage{lineno} % add
\providecommand{\tightlist}{%
  \setlength{\itemsep}{0pt}\setlength{\parskip}{0pt}}

\usepackage{graphicx}
\usepackage{booktabs} % book-quality tables
%%%%%%%%%%%%%%%% end my additions to header

\usepackage[T1]{fontenc}
\usepackage{lmodern}
\usepackage{amssymb,amsmath}
\usepackage{ifxetex,ifluatex}
\usepackage{fixltx2e} % provides \textsubscript
% use upquote if available, for straight quotes in verbatim environments
\IfFileExists{upquote.sty}{\usepackage{upquote}}{}
\ifnum 0\ifxetex 1\fi\ifluatex 1\fi=0 % if pdftex
  \usepackage[utf8]{inputenc}
\else % if luatex or xelatex
  \usepackage{fontspec}
  \ifxetex
    \usepackage{xltxtra,xunicode}
  \fi
  \defaultfontfeatures{Mapping=tex-text,Scale=MatchLowercase}
  \newcommand{\euro}{€}
\fi
% use microtype if available
\IfFileExists{microtype.sty}{\usepackage{microtype}}{}
\usepackage[left=3cm,right=3cm,top=3cm,bottom=3cm]{geometry}
\bibliographystyle{elsarticle-harv}
\usepackage{longtable}
\ifxetex
  \usepackage[setpagesize=false, % page size defined by xetex
              unicode=false, % unicode breaks when used with xetex
              xetex]{hyperref}
\else
  \usepackage[unicode=true]{hyperref}
\fi
\hypersetup{breaklinks=true,
            bookmarks=true,
            pdfauthor={},
            pdftitle={Edge effects and optimal field sizes in Alberta},
            colorlinks=false,
            urlcolor=blue,
            linkcolor=magenta,
            pdfborder={0 0 0}}
\urlstyle{same}  % don't use monospace font for urls

\setcounter{secnumdepth}{5}
% Pandoc toggle for numbering sections (defaults to be off)

% Pandoc citation processing
\newlength{\csllabelwidth}
\setlength{\csllabelwidth}{3em}
\newlength{\cslhangindent}
\setlength{\cslhangindent}{1.5em}
% for Pandoc 2.8 to 2.10.1
\newenvironment{cslreferences}%
  {}%
  {\par}
% For Pandoc 2.11+
\newenvironment{CSLReferences}[3] % #1 hanging-ident, #2 entry spacing
 {% don't indent paragraphs
  \setlength{\parindent}{0pt}
  % turn on hanging indent if param 1 is 1
  \ifodd #1 \everypar{\setlength{\hangindent}{\cslhangindent}}\ignorespaces\fi
  % set entry spacing
  \ifnum #2 > 0
  \setlength{\parskip}{#2\baselineskip}
  \fi
 }%
 {}
\usepackage{calc} % for calculating minipage widths
\newcommand{\CSLBlock}[1]{#1\hfill\break}
\newcommand{\CSLLeftMargin}[1]{\parbox[t]{\csllabelwidth}{#1}}
\newcommand{\CSLRightInline}[1]{\parbox[t]{\linewidth - \csllabelwidth}{#1}}
\newcommand{\CSLIndent}[1]{\hspace{\cslhangindent}#1}

% Pandoc header
\makeatletter \def\ps@pprintTitle{  \let\@oddhead\@empty  \let\@evenhead\@empty  \def\@oddfoot{\centerline{\thepage}} \let\@evenfoot\@oddfoot} \makeatother \usepackage{float} \floatplacement{figure}{H} \newcommand{\beginsupplement}{\setcounter{table}{0} \renewcommand{\thetable}{S\arabic{table}} \setcounter{figure}{0} \renewcommand{\thefigure}{S\arabic{figure}}} \usepackage{setspace} \linenumbers



\begin{document}
\begin{frontmatter}

  \title{Edge effects and optimal field sizes in Alberta}
    \author[University of Calgary]{Samuel V. J. Robinson\corref{1}}
   \ead{samuel.robinson@ucalgary.ca} 
    \author[University of Calgary]{Lan H. Nguyen}
   \ead{hoanglan.nguyen@ucalgary.ca} 
    \author[University of Calgary]{Paul Galpern}
   \ead{paul.galpern@ucalgary.ca} 
      \address[University of Calgary]{2500 University Drive NW, Calgary, AB}
      \cortext[1]{Corresponding Author}
  
  \begin{abstract}
  Abstract goes here
  \end{abstract}
   \begin{keyword} beetles; spiders; harvestmen;\end{keyword}
 \end{frontmatter}

\newpage
\doublespacing

\hypertarget{introduction}{%
\section{Introduction}\label{introduction}}

\begin{itemize}
\tightlist
\item
  Preserving SNL around fields is important for agriculture and conservation

  \begin{itemize}
  \tightlist
  \item
    Ecosystem services
  \item
    Act as reservoirs for beneficial insects
  \item
    Microclimate zones
  \item
    Poorly studied in North America
  \item
    Studies tend to be limited in scope (e.g.~only certain organisms/crops) and applicability
  \end{itemize}
\item
  Ecosystem services can influence both the mean and variability of yield in agroecosystems

  \begin{itemize}
  \tightlist
  \item
    Typically only averages (means) are considered, but higher stability (lower variance) in yield is also valuable
  \end{itemize}
\item
  Field size and field boundaries are directly related to the conservation of SNL in agroecosystems

  \begin{itemize}
  \tightlist
  \item
    Size examples: wheat study from the UK
  \item
    Boundary examples: flower strip studies, hedgerows
  \item
    In North America, large financial incentives to make fields large and homogeneous, especially with large harvesting and planting equipment
  \item
    Crop edge effects cause low yields at the margins of fields because of late emergence, poor microclimate, and competition with weeds
  \item
    Ecosystem services should decay with distance from edge, so crops at the centre of a large field will not benefit from ecosystem services
  \item
    Therefore, there should be a ``goldilocks" field size, where negative edge effects are canceled out by ecosystem services
  \end{itemize}
\item
  Precision yield data holds enormous promise for agronomy

  \begin{itemize}
  \tightlist
  \item
    Limited because of:
  \item
    Lack of standardized formats
  \item
    Yearly calibration data
  \item
    Clear statistical protocols
  \end{itemize}
\end{itemize}

\hypertarget{methods}{%
\section{Methods}\label{methods}}

\hypertarget{data-collection}{%
\subsection{Data collection}\label{data-collection}}

\begin{itemize}
\tightlist
\item
  Precision yield data were collected directly from farmers across Alberta

  \begin{itemize}
  \tightlist
  \item
    Farmers were solicited for yield data through local agronomists, and we received data from a total of X growers across a total of X years
  \item
    Most fields represented only a single year of data, but we did sometimes receive multiple years of data from the same field
  \item
    File formats vary depending on the brand of combine, so we converted data to a standard format (csv) using Ag Leader SMS
  \item
    In total, we analyzed yield data from X field-years of data, containing a total of X million data points
  \end{itemize}
\item
  Yield data was collected in discrete rectangles of the same length as the data interval (distance = speed \(\times\) interval, typically 1 second) and the same width as the combine header (5-7 m)

  \begin{itemize}
  \tightlist
  \item
    We extracted the size of each polygon (m\(^2\)), dry yield (tonnes), and the spatial location, and the sequence of collection (1 - end of harvest)
  \item
    Because of the large number of yield rectangles per field (30-800 thousand), we used the centroid of each polygon as its location
  \item
    Seeding and application rates were constant with each field, so we did not consider inputs in our analysis
  \end{itemize}
\item
  Field boundaries were automatically digitized, then manually checked using satellite imagery and classified land cover data

  \begin{itemize}
  \tightlist
  \item
    Crop boundaries are flexible, and often change from year to year depending on planting and emergence conditions (e.g.~flooding during some years)
  \item
    Additionally, seminatural features often change from year to year

    \begin{itemize}
    \tightlist
    \item
      Ephemeral wetlands are flooded during some years, but consist mainly of grasses during dry years
    \item
      Grass boundaries can change if fields are used for as haying or pasture during crop rotation
    \end{itemize}
  \item
    This makes accurate and consistent classification of field boundaries very difficult
  \item
    Because of this, we used the following general categories for field boundaries:

    \begin{enumerate}
    \def\labelenumi{\arabic{enumi}.}
    \tightlist
    \item
      Standard: thin (\(<10\) m wide) grassy field boundary, often grassy road right-of-ways
    \item
      Wetland: permanent wetland, whose borders are largely unchanged from year-to-year
    \item
      Forest/shrub: permanent windbreaks or remnant forests
    \item
      Grass: larger grassy area (pasture or permanent seminatural grassland)
    \item
      Other crop: different crop with little or no visible boundary between planted areas
    \item
      Bare: unplanted (fallow), flooded area, or unpaved roads
    \end{enumerate}
  \end{itemize}
\end{itemize}

\hypertarget{analysis}{%
\subsection{Analysis}\label{analysis}}

\begin{itemize}
\tightlist
\item
  At each field, we fit an additive model of the effect of boundary distance on crop yield while accounting for within-field spatial variation and temporal variation in the combine yield monitor

  \begin{itemize}
  \tightlist
  \item
    Crop yield varies within a field due to soil conditions, moisture, seeding rates, herbicide application, and previous agricultural practices such as strip farming
  \item
    While sensor calibration can reduce combine-level bias (such as a combine recording consistently higher/lower yields), this does not address sensor drift over time that occurs within fields
  \item
    The yield monitors may record lower yields as sensors accumulate debris during harvest (pers. comm. Trent Clark), leading to changes in accuracy and bias over time
  \item
    Additionally, ground speed is known to be extremely important to yield monitor accuracy (Arslan \& Colvin 2002)
  \item
    To address this, we fit the following model:
  \end{itemize}

  \begin{equation}
  ln(yield) 
  \end{equation}
\end{itemize}

\hypertarget{results}{%
\section{Results}\label{results}}

Results here

\hypertarget{discussion}{%
\section{Discussion}\label{discussion}}

Discussion here

\hypertarget{authors-contributions}{%
\section{Authors' contributions}\label{authors-contributions}}

Author's contribution

\hypertarget{acknowledgements}{%
\section{Acknowledgements}\label{acknowledgements}}

Funding for this research was provided by Ducks Unlimited Canada's Institute for Wetland and Waterfowl Research, the Alberta Canola Producers Commission, Manitoba Canola Growers Association, SaskCanola, the Alberta Biodiversity Monitoring Institute, and the Alberta Conservation Association.

\hypertarget{references}{%
\section*{References}\label{references}}
\addcontentsline{toc}{section}{References}

\hypertarget{refs}{}
\begin{CSLReferences}{1}{0}
\leavevmode\hypertarget{ref-arslan2002}{}%
Arslan, S. \& Colvin, T.S. (2002). An evaluation of the response of yield monitors and combines to varying yields. \emph{Precision Agriculture}, 3, 107--122.

\end{CSLReferences}

\newpage

\hypertarget{appendix-a-supplementary-material}{%
\section*{Appendix A: Supplementary Material}\label{appendix-a-supplementary-material}}
\addcontentsline{toc}{section}{Appendix A: Supplementary Material}

Supplemental materials here


\end{document}

